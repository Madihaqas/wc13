\documentclass[a4paper, addpoints]{exam}

\usepackage{amsfonts,amsmath,amsthm}
\usepackage{framed}
\usepackage[a4paper]{geometry}

\header{CS/MATH 113}{WC13: Structural Induction}{Spring 2024}
\footer{}{Page \thepage\ of \numpages}{}
\runningheadrule
\runningfootrule

\printanswers

\qformat{{\large\bf \thequestion. \thequestiontitle}\hfill}
\boxedpoints

\theoremstyle{definition}
\newtheorem{definition}{Definition}
\theoremstyle{claim}
\newtheorem{claim}{Claim}

\title{Weekly Challenge 13: Structural Induction}
\author{CS/MATH 113 Discrete Mathematics}
\date{Spring 2024}

\begin{document}
\maketitle

\begin{questions}
\titledquestion{$k$-ary tree}[10]
  Definition 5 in Section 5.3 of our textbook defines a \textit{full binary tree}. We extend this definition to a \textit{full $k$-ary tree} as follows.
  \begin{framed}
    \begin{definition}[Full $k$-ary tree]$\null$
      
      \underline{Basis Step} There is a full $k$-ary tree consisting only of a single vertex $r$.
      
      \underline{Recursive Step}  If $T_1,T_2, T_3,\ldots,T_k$ are disjoint full $k$-ary trees, there is a full $k$-ary tree, denoted by $T_1\cdot T_2\cdot T_3\cdot\ldots\cdot T_k$, consisting of a root $r$ together with edges connecting the root to each of the roots of $T_1,T_2, T_3,\ldots,T_k$.
    \end{definition}
  \end{framed}
  We also introduce the following definitions of nodes in a tree.
  \begin{definition}[Leaf node]
    A leaf node in a tree is a node that has no children.
  \end{definition}
  \begin{definition}[Internal node]
    An internal node in a tree is a node that is not a leaf node.
  \end{definition}

  Use structural induction to prove the following claim.
  \begin{claim}
    The number of internal nodes in a full $k$-ary tree with $n$ leaves is $\frac{n-1}{k-1}$.
  \end{claim}
  \begin{solution}
    
    Basis step:\\
    For a full k-ary tree with a single vertex, $\frac{n-1}{k-1}$,\\
    $\frac{1-1}{1-1}$ = 0 which is true as there are no internal nodes present in a k-ary tree with a single leaf.\\
    Recursive step:\\
    Each disjoint Tree $T_1,T_2, T_3,\ldots,T_k$ from the same root r has $\frac{n-1}{k-1}$ internal nodes.\\
    $\frac{n-1}{k_1-1}$, $\frac{n-1}{k_2-1}$, $\frac{n-1}{k_3-1}$, $\dots$, $\frac{n-1}{k-1}$.\\
    Writing in summation form: $\sum_{i=1}^{k} a_i = \frac{k_1 - 1}{n-1} + \frac{k_2 - 1}{n-1} + \frac{k_3 - 1}{n-1} + \dots + \frac{k-1}{n-1}$ which completes the recursive step.\\
    The root r of the disjoint tree is also an internal node, so adding 1 in the formula above,\\
    $\sum_{i=1}^{k} a_i + 1$ will give the number of internal node in a full k-ary tree.

  \end{solution}
\end{questions}

\end{document}
%%% Local Variables:
%%% mode: latex
%%% TeX-master: t
%%% End:
